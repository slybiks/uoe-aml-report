We explored several dimensionality reduction methods with clustering algorithms on the Spotify audio characteristics dataset. We concluded that the best results were obtained using K-Means with PaCMAP. We also tried extending the work partially to cluster songs based on their trajectories. The results are included in Appendix \ref{appendix:F}. We could evaluate the system with the help of other variations of K-Means like MiniBatchKMeans and weighted K-Means, or perhaps some agglomerative clustering methods. Currently we have the artists' song cluster signature on which we can run other clustering methods to group similar artists. Performing region-based clustering of artists to facilitate artist collaboration is one of the primary applications of this approach since different geographic areas around the world produce different types of sounds. Another task would be to train deep learning models like LSTMs to predict the trends of songs. 

